\documentclass[a4paper]{article}
\usepackage[top=30pt,bottom=30pt,left=48pt,right=46pt]{geometry}
\usepackage{graphicx} % Required for inserting images
\usepackage[utf8]{inputenc} % Suporte a caracteres UTF-8
\usepackage[T1]{fontenc}    % Suporte a acentuação
\usepackage[portuguese]{babel} % Língua portuguesa
\usepackage{listings}       % Para formatação de código
\usepackage{xcolor}         % Para cores no código
\usepackage{a4wide}
\usepackage{indentfirst}
\usepackage{hyperref}
\usepackage{algorithm}
\usepackage{algpseudocode}
\usepackage{amsmath}
\usepackage{amssymb}

\lstdefinestyle{python}{
    language=Python,
    showstringspaces=false,
    basicstyle=\ttfamily\small,
    keywordstyle=\color{blue}\bfseries,
    stringstyle=\color{orange},
    commentstyle=\color{green!50!black},
    numbers=left,
    numberstyle=\tiny,
    stepnumber=1,
    numbersep=10pt,
    frame=single,
    breaklines=true,
    captionpos=b,
    tabsize=4
}

\title{Relatório do trabalho de Linguagem de Programação}
\author{Dilmar Aparecido Castanheiro \\
Bernardo Quintella \\ Matheus Constantin
}
\date{December 2024}

\begin{document}

\maketitle

\section{Introdução}

O nosso projeto é um jogo do estilo Run’n Gun, inspirado em clássicos como \textit{Contra} e \textit{Metal Slug}, onde o personagem pode andar, pular, atirar para os lados, recarregar, morrer caso a vida acabe, e o seu objetivo é "derrotar todos os inimigos, ultrapassar os desafios da fase e alcançar o final". Ao atingir o bloco verde, o jogador deve pressionar para baixo, "encerrando a partida com sucesso".

\section{Design}

Nosso jogo possui uma única fase, onde existem 3 tipos de inimigos. O nível é composto por "cinco andares conectados por plataformas, criando desafios verticais e horizontais". O personagem deve descer os andares enfrentando obstáculos e inimigos.

Os inimigos são:
\begin{itemize}
    \item AR: Portam um fuzil, "realizam patrulhas simples" e atiram quando veem o player.
    \item Sniper: Portam uma sniper, "são estrategicamente posicionados" e atiram quando veem o player.
    \item Bazooka: Portam uma bazooka, "representam grande ameaça com explosões" e atiram quando veem o player.
\end{itemize}

\section{Gameplay e Mecânicas}

O personagem possui movimentos básicos como andar, pular e atirar, além de mecânicas mais avançadas, como "recarregar a arma e utilizar estratégias para desviar de tiros inimigos". Os inimigos apresentam comportamentos distintos, "oferecendo variedade e aumentando a complexidade conforme o jogador avança". A fase é estruturada de forma linear, incentivando o jogador a descer andares e enfrentar inimigos.

\section{Aspectos Visuais e Sonoros}

Os gráficos do jogo seguem uma estética pixelada, devido aos sprites utilizados, "inspirando-se em jogos clássicos e criando um apelo nostálgico". Em relação ao som, há "efeitos sonoros dinâmicos que acompanham as ações, como tiros, pulos e impactos", aprimorando a imersão.

\section{Divisão de Tarefas}

Dividimos o desenvolvimento do jogo da seguinte forma: Matheus ficou responsável pelo menu e design, Dilmar fez o relatório, e cuidou do mapa e jogabilidade, Bernardo fez os testes, colisões e inimigos e todos participaram colaborativamente na implementação do código principal.

\section{Organização do Código}

O código está dividido em 10 arquivos principais dentro da pasta \textit{src}, conforme detalhado a seguir:

\begin{itemize}
    \item \textbf{entity.py}: contém a classe responsável pelas entidades do jogo (jogador, inimigos, bala).
    \item \textbf{player.py}: contém a classe responsável pelo jogador, com métodos para "executar ações como atirar, pular e recarregar".
    \item \textbf{enemy.py}: contém as classes responsáveis pelos inimigos, como o sniper, bazooka e AR.
    \item \textbf{bullet.py}: contém a classe responsável pela bala, que "gera dano ao colidir com alvos ou desaparece ao sair do cenário".
    \item \textbf{map.py}: contém o código responsável pelo mapa.
    \item \textbf{menu.py}: contém a classe responsável pelo menu.
    \item \textbf{settings.py}: contém as configurações principais, como "valores padrão para vida, munição e comportamento dos inimigos".
    \item \textbf{game.py}: cria o jogo a partir das classes contidas nos outros arquivos.
    \item \textbf{main.py}: executa o jogo.
    \item \textbf{util.py}: carrega algumas funções úteis.
\end{itemize}

\section{Testes e Melhorias}

Realizamos testes de jogabilidade para garantir que o jogo estivesse balanceado, identificando ajustes necessários em dificuldade, IA dos inimigos e responsividade dos controles. Além disso, recebemos feedback de "usuários externos que testaram o jogo", o que nos ajudou a refinar a experiência final.

\section{Desafios Enfrentados}

Durante o desenvolvimento, enfrentamos desafios como "implementar comportamentos inteligentes para os inimigos" e "resolver problemas de colisão das balas com precisão", mas conseguimos ultrapassar essas dificuldades.

\end{document}
